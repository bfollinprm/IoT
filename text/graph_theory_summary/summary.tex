\documentclass{article}
\usepackage{amsthm}
\usepackage{amsmath}

\def\be{\begin{equation*}}
\def\ee{\end{equation*}}

\title{Useful Network Theory??}
\author{Brent Follin}
\begin{document}
\maketitle

\section{Problem Summary} % it's ok for it to not take any arguments
The general scope of the problem is network complexity in the internet of things (IoT). In particular, I'm looking to investigave how complex a system one can develop before the IoT starts to `fail', a concept that will take some definition later. 
\subsection{definitions}
The IoT consists of a set of connected devices $d_{i} \in \left\{D \right\}$ that each can be in one of a multiplicity of states $s^\alpha_i \in \left\{1, ..., m_i\right\}$, such that the total state space of the IoT is 
\be
M \equiv \prod_i m_i,
\ee
e.g. the state $S^\alpha$ of the IoT ranges in labels from $\left\{1, ..., M\right\}$.

The interesting thing is interactions between systems. A particular example is \textit{if-then interactions}, where if some element $d_i$ is in state $s^\alpha_i$ then some other element $d_j$ changes from inital state $s^\alpha_j$ to $s^\beta_j$

\section{} % It can also take an argument
At vero eos et accusamus et iusto odio dignissimos ducimus qui blanditiis 
praesentium voluptatum deleniti atque corrupti quos dolores et quas 
molestias excepturi sint occaecati cupiditate non provident, similique sunt in 
culpa qui officia deserunt mollitia animi, id est laborum et dolorum fuga. Et 
harum quidem rerum facilis est et expedita distinctio. Nam libero tempore, cum 
soluta nobis est eligendi optio cumque nihil impedit quo minus id quod maxime 
placeat facere possimus, omnis voluptas assumenda est, omnis dolor repellendus.
 Temporibus autem quibusdam et aut officiis debitis aut rerum necessitatibus 
saepe eveniet ut et voluptates repudiandae sint et molestiae non recusandae. 
Itaque earum rerum hic tenetur a sapiente delectus, ut aut reiciendis 
voluptatibus maiores alias consequatur aut perferendis doloribus asperiores repellat.
\end{document}